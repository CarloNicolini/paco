\documentclass[11pt,%                        % primary font size
              a4paper,%                     % A4
]{article}
\usepackage[english]{babel}
\usepackage[utf8x]{inputenc}
\usepackage[T1]{fontenc}
\usepackage{amsmath,amsfonts,amssymb,amsthm}
\usepackage{lettrine}
\usepackage[sc]{mathpazo} % Use the Palatino font

\usepackage{microtype} % Slightly tweak font spacing for aesthetics
\usepackage{graphicx}
% \usepackage{hyperref}
\usepackage{clrscode}
\usepackage{minted}
\usepackage{hyperref}
\usepackage{url}
\usepackage{geometry}{right=1cm,left=1cm}

\title{PACO Partitioning Cost Optimization\\User Manual}

\author{Carlo Nicolini}

\begin{document}
\maketitle
\tableofcontents

\section{Compilation Instructions}

Paco is a tool to build solvers and optimize quality functions on graphs
for community detection. Paco comes as both a C++ library and a support
for Matlab/Octave mex function and additionally as Python package.


\subsection{Software requirements}
\begin{itemize}
\item A recent c++ compiler (PACO is tested on \texttt{g++>4.4} and \texttt{clang}).
 \item \texttt{igraph} libraries at version \texttt{>0.7.1} \url{http://igraph.org/c}.
\item CMAKE \texttt{>3.0} \url{http://cmake.org}.
\end{itemize}

As PACO needs the \texttt{C} libraries \texttt{igraph}, you need to compile them by yourself, as the latest packaged version on Ubuntu is too old.
To compile extract the \texttt{paco\_latest.tar.gz} archive somewhere on your computer and do:

\begin{minted}[bgcolor=gray!20]{bash}
$> cd paco
$> mkdir build
$> cd build
$> cmake ..
\end{minted}

If you want to compile \texttt{paco\_mx} mex function, you need to
enable the \texttt{MATLAB\_SUPPORT} flag. At the last step:

\begin{minted}[bgcolor=gray!20]{bash}
$> cmake -DMATLAB_SUPPORT=True ..
\end{minted}

or equivalently for Octave

\begin{minted}[bgcolor=gray!20]{bash}
$> cmake -DOCTAVE_SUPPORT=True ..
\end{minted}

and then compile the source with \texttt{make}:

\begin{minted}[bgcolor=gray!20]{bash}
$> make
\end{minted}

\subsection{Obtaining igraph in OSX}\label{obtaining-igraph-in-osx}

We strongly suggest to obtain \texttt{igraph} in OSX using
\textbf{Homebrew}.

\begin{minted}[bgcolor=gray!20]{bash}
$> ruby -e "$(curl -fsSL https://raw.githubusercontent.com/
  Homebrew/install/master/install)"
$> brew tap homebrew/science
$> brew install igraph
\end{minted}

\subsection{Obtaining igraph in Ubuntu 12.04 or
greater}

We suggest compiling \texttt{igraph} from source, since PACO needs
\texttt{igraph}\textgreater{}=0.7.1.

Go to \url{www.igraph.org/c} and follow the documentation for the compilation using the autogen script they provide, otherwise follow the following instructions:

\subsubsection{Compilation and installation of igraph on Ubuntu 12.04 or
greater}\label{compilation-and-installation-of-igraph-on-ubuntu-12.04-or-greater}

Download igraph-0.7.1 source code:

\begin{minted}[bgcolor=gray!20]{bash}
$> cd
$> wget http://igraph.org/nightly/get/c/igraph-0.7.1.tar.gz
\end{minted}
Extract the code under your home folder:

\begin{minted}[bgcolor=gray!20]{bash}
$> tar -zxvf igraph-0.7.1.tar.gz
\end{minted}
Install the necessary libs, under Ubuntu:

\begin{minted}[bgcolor=gray!20]{bash}
$> sudo apt-get install build-essential libxml2-dev zlib1g-dev
\end{minted}
Try to configure and compile igraph:

\begin{minted}[bgcolor=gray!20]{bash}
$> cd igraph-0.7.1/
$> ./configure
$> make 
\end{minted}
If you have many processors on your computer, make in parallel mode,  instead of \texttt{make} just issue \texttt{make\ -j8} for example to   compile with 8 cores.
Check if installation went fine:

\begin{minted}[bgcolor=gray!20]{bash}
$> make check
\end{minted}

If tests are passed, then install it. I usually install \emph{igraph}  under \texttt{/usr/lib/}. To do that I call \texttt{make\ install} with administrator privileges.

\begin{minted}[bgcolor=gray!20]{bash}
$> sudo make install
\end{minted}
If you don't have administrator privileges you can compile igraph under your home folder.
In the rest of the notes we assume that the Igraph include directory is
\texttt{/usr/local/include/igraph}.

\section{Windows support:}\label{windows-support}

Despite everything should be ready to be ported happily in Windows, I
don't have time to let the code compile smoothly on Windows. If you want
to join me in writing PACO for Windows let me know.

\section{Frequently asked questions}

\subsection{QUESTION: Linking libstdc++.so.6
problem}\label{linking-libstdc.so.6-problem}

I can compile Paco for MATLAB but after calling \texttt{paco\_mx},
MATLAB prompts me with the following error message:

\begin{minted}[bgcolor=gray!20]{bash}
Invalid MEX-file '~/paco_mx.mexa64': /usr/local/MATLAB/R2015a/
   bin/glnxa64/../../sys/os/glnxa64/libstdc++.so.6: version
   `GLIBCXX_3.4.21' not found (required by ...
\end{minted}

This problem means that the \texttt{libstdc++.so.6} inside the Matlab
library folder is pointing to a version of \texttt{libstdc++} older than
the system one, usually stored in \texttt{/usr/lib/x86\_64} folder.

To solve the issue you need to redirect the symbolic links in the MATLAB
folder to the systemwise \texttt{libstdc++}. Hereafter we assume the
MATLAB folder to be \texttt{/usr/local/MATLAB/R2015a} and the system to
be some Linux variant.

Two of the symlinks for libraries need to be changed:

\begin{minted}[bgcolor=gray!20]{bash}
$> cd /usr/local/MATLAB/R2015a/sys/os/glnxa64
$> ls -l
\end{minted}
The sym links for libstdc++.so.6 and libgfortran.so.3 should point to
versions in /usr/lib, not local ones.

Before changing this libraries, first make sure \texttt{g++-4.4} and
\texttt{libgfortran3}are installed :

\begin{minted}[bgcolor=gray!20]{bash}
$> sudo apt-get install g++-4.4 libgfortran3
\end{minted}
Now, modify the symlinks:

\begin{minted}[bgcolor=gray!20]{bash}
$> sudo ln -fs /usr/lib/x86_64-linux-gnu/libgfortran.so.3.0.0 libgfortran.so.3
$> sudo ln -fs /usr/lib/gcc/x86_64-linux-gnu/4.4/libstdc++.so libstdc++.so.6
\end{minted}

This command makes the \texttt{libstdc++.so.6} point to the
\texttt{g++-4.4} \texttt{libstdc++} library.

For other informations take a look at
\url{http://dovgalecs.com/blog/matlab-glibcxx\_3-4-11-not-found/} or 
\url{https://github.com/RobotLocomotion/drake/issues/960} which are very
similar problems.

\section{PACO Library}
do this

\section{Command line}
\begin{minted}[bgcolor=gray!20]{bash}
./paco_optimizer karate.gml
\end{minted}
\section{MATLAB}
Here is an example on how to use PACO for MATLAB:

\begin{minted}[bgcolor=gray!20]{matlab}
addpath('YOUR_PATH_WHERE_PACO_MEX_IS_STORED');
% now load the karate club graph,a  34x34 binary adjacency matrix
A=load('~/karate.adj'); % load the karate club graph
[membership, quality] = paco_mx(A); % Optimize discrete Surprise
\end{minted}


\section{Octave}
\begin{minted}[bgcolor=gray!20]{matlab}
addpath('YOUR_PATH_WHERE_PACO_MEX_IS_STORED');
% now load the karate club graph,a  34x34 binary adjacency matrix
A=load('~/karate.adj'); % load the karate club graph
[membership, quality] = paco_oct(A); % Optimize discrete Surprise
[membership, quality] = paco_oct(A,'quality',2); % Optimize Asymptotical Surprise
[membership, quality] = paco_oct(A,'quality',2,'nrep',5); % Optimize Asymptotical Surprise, do 5 repetitions
\end{minted}


\section{MATLAB}
\begin{minted}{matlab}
A=zeros(10,10);
}
\end{minted}
\section{Python}
\section{MATLAB}
Example: passing graph as adjacency matrix
\begin{minted}[bgcolor=gray!20]{python}

import numpy as np
from pypaco import paco
G = nx.karate_club_graph()
A  = nx.to_numpy_matrix(G)
[membership,quality] = paco(A, quality=0, nreps=10)
\end{minted}  
Example: passing graph as edges list

\begin{minted}[bgcolor=gray!20]{python}
import numpy as np
from pypaco import paco
G = nx.karate_club_graph()
E = np.array(G.edges()).astype(float64) # E must be a numpy array of doubles, not a list
[membership,quality] = paco(E, quality=0, nreps=10)
\end{minted}

Example: passing graph as weighted edges list

\begin{minted}[bgcolor=gray!20]{python}    
import numpy as np
from pypaco import paco
G = nx.karate_club_graph()
E = np.array(G.edges())
# Generate weights on the edges, in values between 0 and 10-1
W = np.random.randint(10,size=[G.number_of_edges(),1])
EW = np.concatenate((E,W),axis=1).astype(float)
[membership,quality] = paco(EW, quality=2, nreps=10)
\end{minted}



\end{document}